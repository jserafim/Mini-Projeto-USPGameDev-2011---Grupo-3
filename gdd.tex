\documentclass[12pt, a4paper]{article}

\usepackage[portuguese]{babel}
\usepackage[utf8]{inputenc}
\usepackage{times}

\usepackage{float}

\begin{document}

\thispagestyle{empty}
    \begin{center}
        \textbf{\LARGE{Universidade de São Paulo}}
        \vspace*{0.5 cm}

        {\LARGE{USPGameDev}}
        
        \vspace*{3.0 cm}
        \textbf{\large{Galaxy Warrior\\
                \emph{Game Design Document}}}
    \end{center}

    \vskip 3cm

    \begin{flushright}
        Mini-projeto 2011

        \vskip 3cm

    \end{flushright}
    \vskip 3.5cm

    \begin{center}
        Ana Julia Travia\\
        Caio Ossamu Honda Tokutake\\
        Felipe Yamaguti\\
        Jefferson Serafim Ascaneo\\
        Stefan Corsini Radkowski
        \vskip 1cm
        São Paulo, Brasil\\
        26 de janeiro de 2011
    \end{center}
\pagebreak

\tableofcontents
\pagebreak

\section{Sinopse}
    Em \emph{Galaxy Warrior}, você controla Yvan, um jovem russo que se encontra
    no contexto da Segunda Guerra Mundial, lutando ao lado de seu amigo Lazlo.
    Durante uma missão, ele é abduzido e entra em contato com uma raça alienígena,
    os Vassodorfs, que num primeiro momento parecem ser os inimigos.
    
    Yvan deve buscar seu amigo Lazlo, que também foi abduzido, mas acaba se 
    envolvendo em uma trama intergaláctica da qual também participa uma outra 
    raça, os Givurths. 

\section{Gênero}
    \emph{Galaxy Warrior} é um jogo de plataforma, limitando o jogador a se movimentar
    em duas dimensões. As fases são compostas de diversas camadas, sendo possível
    transitar entre elas através de portas. Com isso, é possível traçar diversos 
    percursos dentro de cada fase.
    
    O personagem principal possui poderes que o ajudam
    a interagir com o cenário e resolver os diversos puzzles que são apresentados.
    O jogador também pode escolher não utilizar seus poderes, enfrentando os inimigos
    e solucionando os puzzles apenas com sua força e armas. Com isso, o jogo pode se
    focar mais na ação ou mais nos puzzles, dependendo das escolhas do jogador.

\section{Enredo}
    Yvan está lutando contra os nazistas na Segunda Guerra Mundial. Após matar
    os inimigos, reconquistando o território soviético, é abduzido e acorda
    na prisão de uma nave. Percebe algo de diferente em si mesmo, pois consegue
    mover objetos com o pensamento, habilidade que utiliza para escapar.

    Tentando sobreviver, vai enfrentando os estranhos soldados e criaturas inimigas,
    conseguindo novas armas e equipamentos pelo caminho. Descobre que o seu amigo 
    Lazlo também foi capturado e vai em sua busca. Depara-se com o primeiro chefe
    da fase, na verdade um sub-chefe, e consegue derrotá-lo após receber a ajuda
    de Golok. Continua percorrendo a nave, e enfrenta o chefe. Quase é derrotado
    durante a batalha, mas é resgatado pelos aliens da raça B.

    Os aliens da raça B liberam novos poderes em Yvan, e o colocam contra a raça A
    com suas mentiras. Finjem ajudá-lo enviando-o para o planeta natal da raça A,
    para combatê-los e resgatar seu amigo. Neste novo planeta, Yvan luta contra
    novos e mais poderosos inimigos, mas acaba não conseguindo chegar a tempo para
    salvar seu amigo Lazlo, que é assassinado enquanto ele assiste sem poder fazer
    nada. Com o impacto do ocorrido, tem um ataque psicótico.

    Um ser da raça B o salva de seus delírios e o leva para um lugar seguro, o Forte
    da resistência. Lá, Yvan decide vingar a morte de seu amigo de infância, e
    parte em sua missão percorrendo o subterrâneo do planeta. Encontra uma nave na
    qual consegue se infiltrar, e que está indo para o sistema solar onde está
    o planeta natal da raça B.

    Após tomar o controle da nave e aterrisá-la no planeta da raça B, explora
    a superfície do mesmo em busca do líder desta raça. Por todo o caminho, enfrenta
    outras raças controladas pela raça B, e começa a se perguntar onde estariam seus
    verdadeiros inimigos. Dentro da capital, descobre pistas de seu paradeiro, mas
    ao chegar no local onde estariam, encontra seu amigo Lazlo, que acreditava estar
    morto. Descobre que agora ele está controlando toda a raça B, e fica sabendo de
    seus planos maquiavélicos. Ainda abalado pelo choque, resolve enfrentar Lazlo.

    Após derrotá-lo, parte em busca dos outros humanos que também fizeram parte
    das experiências da raça B, e que agora estão espalhados pela Galáxia.
    Teme o que está por vir, pois sabe que a raça B não estava criando um exército
    em vão. Mas não lhe resta alternativa senão partir atrás das únicas pessoas que ainda
    podem compreendê-lo, agora que seu planeta está arrasado pela
    guerra, e seu melhor amigo, morto por suas próprias mãos.

\section{Enredo 2}
    \subsection{URSS – República do Tatarstão, Kazan - 1927}

	É uma bela tarde junho, Yan Shvarts tem apenas 5 anos, está fazendo
    um picnic com seus pais, o médico Mikhail Shvarts e sra. Anastasiya 
    Shvarts e sua irmãzinha de 2 anos, Yevegeny. No entanto, Yan se perdeu
    de seus pais, depois de perseguir uma linda borboleta. Yan não 
    consegue achar o caminho de volta, já está ficando tarde e ele sente 
    cada vez mais medo, pois teme, como toda criança o escuro, quer
    abraçar sua mãe e se sentir seguro novamente. 
	O jogador deve ajudar Yan encontrar sua mamãe. A jogabilidade nesse 
    momento é afetada, pois a criança tem medo de prosseguir, por isso 
    anda lentamente, sempre olhando para trás. O jogador prossegue sem 
    grandes obstáculos, somente deve pular pedras e buracos, algumas 
    vezes deve desviar de insetos. 
	Após 2 minutos de jogo, aparece um lobo e Yan é encurralado, porém
    um garotinho de 10 anos aparece e o salva da morte. Esse é Lazlo,
    um menino que fora abandonado pelos seus pais, diz saber pouco de
    seu passado, conhece as regras da natureza, vive nas matas no 
    verão e nas cidades no inverno.
	Yan pede para seu novo amigo o ajudar novamente e o levar para casa, 
    para sua mãe. Lazlo promete levá-lo para casa se ele lhe der uma 
    mecha de seu cabelo e ele gostar da seu cheiro. Yan aceita o condição 
    sem estranhá-la, pois estava com muito medo da escuridão que os 
    cercava. Lazlo cheira o cabelo do pequeno, sorri e o leva para casa.
	Seus pais o recebem desesperados, pois estavam procurando o garoto 
    por todo lugar. Dona Anastasiya senti pena de Lazlo e decide 
    adotá-lo. Assim, se inicia uma grande amizade, na qual Yan sempre 
    será agradecido à Lazlo por salvar sua vida.

	\subsection{Volgogrado - 1943}

	Lazlo e Yan lutam lado a lado na Batalha de Stalingrado, matam 
    diversos soldados nazistas. Nesse momento do jogo, o jogador irá 
    aprender a atirar, rastejar e  atirar bombas. Após 3 minutos de 
    jogo, Lazlo e Yan se separam, Yan tenta desviar dos ataques aéreos 
    e Lazlo vai com outros soldados tentar cercar os inimigos. No entanto,
    após 2 minutos de jogo, Lazlo manda um pedido de socorro para Yan, 
    pois eles caíram numa armadilha, muitos soldados foram mortos, ele 
    era um dos únicos sobreviventes, mas havia caído num buraco. Yan tem 
    que resolver um puzzle para conseguir pegar uma corda e prosseguir 
    no jogo. Yan ainda tem que matar uns inimigos para chegar até Lazlo,
    quando chega ao local onde seu amigo está, usa a corda para salvá-lo.
	Yan recebe um medalha de honra por sua coragem e seu resgate.

	\subsection{1945}

	Yan e Lazlo são soldados prestigiados e foram convocados junto
    com uma seleta equipe de outros soldados para um missão secreta,
    que deveria impedir com que os inimigos nazistas roubassem
    informações sigilosas da URSS. Nessa missão deverá haver resolução 
    de dois puzzles, porém quando chegam perto de desmascarar e 
    descobrir os agentes secretos, uma luz clara invade a tela e 
    ambos são abduzidos.

	\subsection{Nave Vassodorf}

	Yan acorda meio atordoado, está acorrentado, se encontra num 
    ambiente estranho, cheio de gosma azul. Não sabe que o lugar 
    que se encontra é uma nave alienígena. Acredita que se encontra
    numa prisão nazista (pensamento). Yan consegue se soltar das 
    correntes utilizando telepatia, poder que adquiriu durante um 
    processo posterior à abdução, do qual não se lembra, portanto 
    acha que as correntes se soltaram sozinhas, porque estavam frouxas
    (pensamento). Pega suas roupas e armas e sai a procura de seus 
    amigos, principalmente de Lazlo.  Estranha o cenário, os 
    computadores, as colunas de plasma, numa delas vê um amigo preso,
    não consegue se comunicar com ele, nem quebrar a coluna, ao lado 
    no computador está escrito: Experimento nº 2478, Vladimir Koskovisk.
    Continua andando na tela, vê um soldado que estava com ele na missão,
    pergunta por Lazlo, porém o soldado descontrolado o ataca, lutam,
    Yan relutante mata seu colega. Yan continua a fase e tem que matar 
    diversos humanos, depois tem que resolver um puzzle e, assim, acaba 
    descobrindo que possui poderes especias. Depois, Yan fica preso numa 
    sala vazia, a qual não fica assim por muito tempo, pois uma geléia 
    alien, aumento cada vez mais de tamanho e Yan não consegue destruí-la,
    quando está para atingi-lo, um alien da raça Vassodorf, chamado
    Golok, o salva. Yan se assusta com seu aspecto monstruoso e fica 
    parado, encarando-o. Golok fala algumas palavras inteligíveis e 
    sai correndo para o lado oposto ao de Yan.  Yan lê algumas 
    informações nas telas dos computadores, descobre que Lazlo é uma
    das cobaias dos experimentos e sai à procura dele. Continua fase,
    enfrenta mais alguns seres humanos, resolve mais um puzzle e 
    enfrente o último chefão, que é um alien da raça Vassodorf, porém 
    diferente de Golok, esse é maior e vermelho. Yan consegue atacá-lo,
    porém seus ataques são muito fracos, por isso quando seu sangue está
    para acabar, ele é resgatado pelos aliens da raça Givurth.

	\subsection{Nave Givurth}

	Yan acorda cercado de seres estranhos, os aliens da raça Givurth.
    Yan tenta escapar, porém o líder dos Givurth, Levian, o acalma e diz
    que explicar o que está acontecendo. Então, Levian explica o processo
    que ele sofreu durante a abdução, diz que ele não consegue mais falar
    russo, e sim a língua extraterrestre universal, por isso havia 
    consigo ler as informações nas telas dos computadores. Explicou 
    também como adquiriu os poderes, pela ativação de genes inativos.
    Diz também que Yan era um ser superior aos seres humanos e que eles
    não podiam deixá-lo morrer, inexperiente, sem conhecer seu 
    verdadeiro potencial numa nave vassodorfiana, seres brutos e 
    estúpidos. Assim, Levian explica a rivalidade entre essas raças e 
    iminência de uma guerra intergalactia, o que fez com que a raça 
    vassodorfiana, controlasse e abduzisse vários seres humanos. Yan 
    pergunta por Lazlo, mas Levian não conhece nenhum humano com esse 
    nome e diz que provavelmente, ele é ainda prisioneiro dos 
    vassodorfianos. Yan quer ir atrás de Lazlo, porém Levian adverte 
    que Yan não está preparado para batalha com seres fortes com os 
    Vassodorf. Por isso, pergunta a Yan se ele quer ir a uma fase de 
    treinamento. Aqui o jogador pode escolher entre ir ou não. Após a 
    fase de treinamento, Yan ainda explora a nave dos Givurth, desvenda 
    alguns puzzles e enfrenta como chefão, Aida, uma humana, que 
    restabeleceu involuntariamente sua conexão com os comandos dos 
    Vassodorf. Yan derrota Aida e ao vencer, não a mata, mas a liberta 
    do controle. Os givurthianos agradecem Yan e o mandaram para o 
    planeta dos Vassodorf, para procurar seu amigo.
	
	\subsection{Planeta dos Vassodorf}

	Yan enfrenta numerosos puzzles, mata vassodorfs que o atacam, 
    desenvolve suas habilidades, basicamente. No meio, encontra Golok, 
    mas Golok foge. Yan percorre o caminho seguinte seguindo Golok, o 
    que o leva a encontrar a sede da resistência vassodorfiana. Então,
    yan descobre que tudo o que ele sabe é uma mentira. São os Givurths
    que fazem os experimentos do mal com genes. Esses membros da 
    resistência são imunes ao virus que modifica os genes. Yan pede 
    ajuda para retornar a nave e tentar encontrar seu amigo Lazlo.

	\subsection{Bunker dos Vassodorfs}

	Yan percorre tuneis mortais e desenvolve mais suas habilidades 
    com diversos puzzles. Yan perde acesso a qualquer tipo de arma 
    fisica e deve se concentrar em desenvolver seus poder psiquicos 
    porque isso irá ajudá-lo na batalha final. Ao terminar a fase 
    ele reencontra Aida, fugindo pelo corredor no sentido contrário. 
    Ela pede pra não se meter.

	\subsection{Nave dos Givurths, pousada no Planeta Vassodorf}
	
	Ele volta para a nave, e Aida o segue, sem ele perceber. Ele 
    consegue resolver puzzles, abrir caminhos antes desconhecidos, 
    tudo porque Aida está o ajudando. Ele reencontrar os aliens Givurths 
    e deve enfrentá-los numa batalha puzzle, onde ele deverá solucionar 
    diversos puzzles. Quando ele termina, há um curto circuito, uma tela 
    aparece no meio do cenário com a imagem de Lazlo, porém há uma forte 
    interferência, a imagem está comprometida e Yan só entende que deve 
    encontrar Lazlo.

	\subsection{Nave dos Givurths, fase de fuga}

	Nesta fase, Yan deve fugir da nave Givurth desmoronando. A tela 
    tem rolagem automática, então Yan deve estar sempre em movimento. 
    Não há puzzles, nem vilões, só obstáculos, alguns que Yan deve 
    ultrapassar com poder Psíquico ou armas. No fim, quando ele ecapa 
    da nave que explode, ele se depara com uma enorme nave mãe.

	\subsection{Nave Mãe}

	Nessa fase, Yan percorre a nave, que só tem puzzles, até chegar 
    a uma sala de comando. Lá está Lazlo com Aida. Lazlo dá um discurso 
    de twist, e mata Aida. Yan então batalha com Lazlo. Uma batalha 
    simples, mas no final, Lazlo escapa. 
	Ocorre mais um puzzle para escapar da sala de comando, com 
    contagem regressiva, pois a sala está se enchendo de gás venenoso.
    Quando ele consegue escapar, chega á sala central da nave mãe, 
    onde ocorre a batalha final com Lazlo, que é bem complexa e envolve 
    vários níveis, cada um explorando uma habilidade desenvolvida 
    durante jogo. No final, o jogador pode escolher dentre três finais: 
    matar Lazlo e restaurar a paz no espaço, não matar Lazlo e dominar 
    o espaço juntos ou matar Lazlo e dominar o espaço em seu lugar. Se 
    o jogador manter Lazlo vivo, invariavelmente Lazlo vai matá-lo.


\section{Personagens}
    \subsection{Yvan}
        Com a recente morte de seu pai, e a falta de alimentos na cidade, Yvan
        decide junto com seu amigo Lazlo defender a nação soviética dos ataques
        nazistas. Já na sua primeira batalha é reconhecido por seus feitos,
        e é enviado, junto com seu amigo e mais alguns homens, para reconsquistar
        e defender um ponto estratégico que foi tomado recentemente.

        Após conseguirem o feito inacreditável e reclamarem o prédio, perdendo
        apenas três homens, um terremoto seguido de um clarão os deixa inconscientes.
        Ao acordar, Yvan percebe ter sido capturado e é mantido preso por 
        criaturas extraterrestres. Seria este algum disfarce do inimigo para
        extrair informações? Que esconderijo e tecnologias estranhas seriam
        aquelas? Após alguns momentos de reflexão, Yvan decide que deve se 
        preocupar apenas em escapar daquele lugar, deixando estas perguntas
        para depois.

    \subsection{Lazlo}
        Filho único, e descendente de uma antiga linhagem de nobres, agora já
        esquecida, Lazlo é o amigo mais próximo de Yvan, conhecendo-o desde a 
        infância. Aficionado por guerras e armas, foi o principal responsável
        por convencer Yvan a lutar no exército.

        Ajudou Yvan em sua última batalha, sendo também capturado ao final dela.
        Porém, seu atual paradeiro é desconhecido.

    \subsection{Golok}
        Pertencente à raça A, Golok faz parte da resistência contra a raça B. Decide
        ajudar Yvan a encontrar seu amigo, e ainda lhe passa informações sobre
        os recentes acontecimentos, pois percebe que Yvan pode ser um importante
        aliado contra a raça B.

    \subsection{Luvian}
        Líder da raça B, resgata Yvan de sua iminente morte e lhe explica
        os motivos da raça A ao atacarem a Terra.
\end{document}
