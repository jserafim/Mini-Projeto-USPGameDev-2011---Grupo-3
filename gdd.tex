\documentclass[12pt, a4paper]{article}

\usepackage[portuguese]{babel}
\usepackage[utf8]{inputenc}
\usepackage{times}

\usepackage{float}

\begin{document}

\thispagestyle{empty}
    \begin{center}
        \textbf{\LARGE{Universidade de São Paulo}}
        \vspace*{0.5 cm}

        {\LARGE{USPGameDev}}
        
        \vspace*{3.0 cm}
        \textbf{\large{Galaxy Warrior\\
                \emph{Game Design Document}}}
    \end{center}

    \vskip 3cm

    \begin{flushright}
        Mini-projeto 2011

        \vskip 3cm

    \end{flushright}
    \vskip 3.5cm

    \begin{center}
        Ana Julia Travia\\
        Caio Ossamu Honda Tokutake\\
        Felipe Yamaguti\\
        Jefferson Serafim Ascaneo\\
        Stefan Corsini Radkowski
        \vskip 1cm
        São Paulo, Brasil\\
        26 de janeiro de 2011
    \end{center}
\pagebreak

\tableofcontents
\pagebreak

\section{Enredo}
    Yvan está lutando contra os nazistas na Segunda Guerra Mundial. Após matar
    os inimigos, reconquistando o território soviético, é abduzido e acorda
    na prisão de uma nave. Percebe algo de diferente em si mesmo, pois consegue
    mover objetos com o pensamento, habilidade que utiliza para escapar.

    Tentando sobreviver, vai enfrentando os estranhos soldados e criaturas inimigas,
    conseguindo novas armas e equipamentos pelo caminho. Descobre que o seu amigo 
    Lazlo também foi capturado e vai em sua busca. Depara-se com o primeiro chefe
    da fase, na verdade um sub-chefe, e consegue derrotá-lo após receber a ajuda
    de Golok. Continua percorrendo a nave, e enfrenta o chefe. Quase é derrotado
    durante a batalha, mas é resgatado pelos aliens da raça B.

\section{Personagens}
    \subsection{Yvan}
        Com a recente morte de seu pai, e a falta de alimentos na cidade, Yvan
        decide junto com seu amigo Lazlo defender a nação soviética dos ataques
        nazistas. Já na sua primeira batalha é reconhecido por seus feitos,
        e é enviado, junto com seu amigo e mais alguns homens, para reconsquistar
        e defender um ponto estratégico que foi tomado recentemente.

        Após conseguirem o feito inacreditável e reclamarem o prédio, perdendo
        apenas três homens, um terremoto seguido de um clarão os deixa inconscientes.
        Ao acordar, Yvan percebe ter sido capturado e é mantido preso por 
        criaturas extraterrestres. Seria este algum disfarce do inimigo para
        extrair informações? Que esconderijo e tecnologias estranhas seriam
        aquelas? Após alguns momentos de reflexão, Yvan decide que deve se 
        preocupar apenas em escapar daquele lugar, deixando estas perguntas
        para depois.

    \subsection{Lazlo}
        Filho único, e descendente de uma antiga linhagem de nobres, agora já
        esquecida, Lazlo é o amigo mais próximo de Yvan, conhecendo-o desde a 
        infância. Aficionado por guerras e armas, foi o principal responsável
        por convencer Yvan a lutar no exército.

        Ajudou Yvan em sua última batalha, sendo também capturado ao final dela.
        Porém, seu atual paradeiro é desconhecido.

    \subsection{Golok}
        Pertencente à raça A, Golok faz parte da resistência contra a raça B. Decide
        ajudar Yvan a encontrar seu amigo, e ainda lhe passa informações sobre
        os recentes acontecimentos, pois percebe que Yvan pode ser um importante
        aliado contra a raça B.

    \subsection{Luvian}
        Líder da raça B, resgata Yvan de sua iminente morte e lhe explica
        os motivos da raça A ao atacarem a Terra.
\end{document}
